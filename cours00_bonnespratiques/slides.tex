\documentclass{beamer}
\usepackage[french]{babel}
\usepackage{hyperref}
\usepackage{graphicx}
\usepackage{amsmath,amssymb}
\usepackage{tabularx}
\usepackage{booktabs}
\usepackage[compatibility=false]{caption}
\usepackage[toc,page]{appendix}
\usepackage{minted}
\usepackage{xspace}
\usepackage{fourier}

\setminted{autogobble}

\setbeamercovered{transparent}
%\makeatletter
%  \def\beamer@calltheme#1#2#3{%
%    \def\beamer@themelist{#2}
%    \@for\beamer@themename:=\beamer@themelist\do
%    {\usepackage[{#1}]{\beamer@themelocation/#3\beamer@themename}}}
%
%  \def\usefolder#1{
%    \def\beamer@themelocation{#1}
%  }
%  \def\beamer@themelocation{}
%
%\patchcmd{\minted@colorbg}{\noindent}{\medskip\noindent}{}{}
%\apptocmd{\endminted@colorbg}{\par\medskip}{}{}
%\makeatother

\newcolumntype{Y}{>{\centering\arraybackslash}X}

%\usefolder{../theme}
\usetheme[numbering=fraction,block=fill,progressbar=frametitle]{metropolis} %Use metropolis theme
%\usetheme{CambridgeUS} %Use Cambridge theme

\definecolor{bg}{rgb}{0.95,0.95,0.95}
\setminted{bgcolor=bg,fontsize=\scriptsize,autogobble,mathescape,breaklines,tabsize=2}
\setmintedinline{breaklines,autogobble,fontsize=\scriptsize}
\setbeamersize{text margin left=8pt,text margin right=8pt}

\begin{document}

\title[Cours d'algorithmique]{Algorithmique et structures de données}
\author[nicolas.audebert@lecnam.net]{Nicolas Audebert}
\setmainfont{Fira Sans}


\AtBeginSection[]{
  \begin{frame}{Plan de la séance}
  \small \tableofcontents[currentsection]
  \end{frame}
}

\date[30 janv. 2017]{Mercredi 30 janvier 2018}
\subtitle{Bonnes pratiques de programmation en C++}
\maketitle

\begin{frame}{Bien travailler en C++}

\begin{block}{Compiler souvent}
    \begin{itemize}
        \item Compiler permet de s'assurer qu'il n'y a pas d'\textbf{erreur de type} (exemple : \mintinline{cpp}{std::string} au lieu d'un \mintinline{cpp}{float})
        \item Compiler permet de s'assurer qu'il n'y a pas d'\textbf{erreur de syntaxe} (exemple : oubli d'un \texttt{;})
        \item Compiler permet de lever des \textbf{avertissements} : variables déclarées mais non utilisées, transtypages douteux, etc.
    \end{itemize}
\end{block}

\begin{block}{Dans son environnement de développement}
\textbf{QtCreator} et \textbf{VisualStudio} permettent de compiler en un clic. Abusez-en.
\end{block}

\end{frame}

\begin{frame}[fragile]{Style de programmation}

\begin{block}{Accolades}
Choisir un style et s'y \textbf{tenir}.
\end{block}

\begin{columns}
    \begin{column}{0.5\textwidth}
    \begin{minted}{cpp}
    // Style K&R
    void a_function(int x, int y){
        if (x == y) {
            something1();
            something2();
        } else {
            somethingelse1();
            somethingelse2();
        }
        finalthing();
    }
    \end{minted}
    \end{column}
    \begin{column}{0.5\textwidth}
    \begin{minted}{cpp}
    // Style Allman
    void a_function(int x, int y)
    {
        if (x == y)
        {
            something1();
            something2();
        }
        else
        {
            somethingelse1();
            somethingelse2();
        }
        finalthing();
    }
    \end{minted}
    \end{column}
\end{columns}
\end{frame}

\begin{frame}[fragile]{Style de programmation}

\begin{block}{Noms de fonctions, variables, classes}
On préférera utiliser la \mintinline{cpp}{CamelCase} pour les classes et la \mintinline{cpp}{snake_case} pour tout le reste.
\end{block}

\begin{exampleblock}{Exemple}
\begin{minted}{cpp}
class Cluster {
  private:
    int r, g, b;
    int nb_points;
  public:
    get_average_color();
};

Color Cluster::get_average_color(){
  return Color(r,g,b);
}
\end{minted}
\end{exampleblock}
\end{frame}

\begin{frame}[fragile]{Style de programmation}
\begin{block}{Écrire du code clair et concis}
Éviter les parenthèses superflues, les variables inutiles, les lignes à rallonge\dots
\end{block}

\begin{columns}
    \begin{column}{0.52\textwidth}
    \begin{minted}{cpp}
    // NON ! Code peu élégant
    bool Cluster::operator==(Cluster C){
        bool est_egal=true;
        if (n!=C.n || r!=C.r ||
            g!=C.g || b!=C.b){
            est_egal=false;
        }
        return est_egal;
    }
    \end{minted}
    \end{column}
    \begin{column}{0.52\textwidth}
    \begin{minted}{cpp}
    // Version plus claire
    bool Cluster::operator==(Cluster C){
        bool egal = n == C.n &&
                    r == C.r &&
                    g == C.g &&
                    b == C.b;
        return egal;
    }

    // On peut même directement écrire 
    bool Cluster::operator==(Cluster C){
        return (n == C.n && r == C.r &&
                g == C.g && b == C.b);
    }
    \end{minted}
    \end{column}
\end{columns}
\end{frame}

\begin{frame}[fragile]{Style de programmation}
\begin{block}{Indentation}
Chaque bloc doit être \textbf{indenté} et si possible mis en valeur par des accolades.
\end{block}

\begin{columns}
    \begin{column}{0.5\textwidth}
    \begin{minted}{cpp}
    // NON
    double maximum(double x, double y){
    double res;
    if (x>y)
    res=x;
    else
        res=y;
        return y;}
    \end{minted}
    \end{column}
    \begin{column}{0.5\textwidth}
    \begin{minted}{cpp}
    // Oui
    double maximum(double x, double y){
        if (x > y) {
            return x;
        } else {
            return y;
        }
    }
    \end{minted}
    \end{column}
\end{columns}
\end{frame}

\begin{frame}[fragile]{Style de programmation}

\begin{block}{Écrire du code lisible}
\begin{itemize}
    \item Indentation correcte (pas de bloc mal indenté)
    \item Noms de variables et de fonctions clairs
    \item Du code \textbf{commenté}
\end{itemize}
\end{block}

\begin{columns}
    \begin{column}{0.4\textwidth}
    \begin{minted}{cpp}
    bool Vector::nul const (){
      if (size()==0)
        return (true);
      int n = size();
      int res = 0;
      for(int i=0; i<n; i++){
        res = (res + tab[i]);
      }
      if (res == 0)
        return true;
      return false;
    }
    \end{minted}
    \end{column}
    \begin{column}{0.6\textwidth}
    \begin{minted}{cpp}
    // Détermine si la somme du vecteur = 0
    // Ne modifie pas le vecteur
    bool Vector::somme_nulle() const {
      // On vérifie que le vecteur a une
      // taille > 0 pour accéder aux valeurs
      int taille = size();
      if taille == 0
          return true;
      int somme = 0;
      // Calcul de la somme des éléments
      for(int i=0; i<taille; i++)
          somme += tab[i];
      return (somme == 0);
    }
    \end{minted}
    \end{column}
\end{columns}
\end{frame}

\end{document}
